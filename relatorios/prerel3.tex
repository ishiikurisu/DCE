\documentclass[12pt, a4paper, twoside]{article}
\usepackage[utf8]{inputenc}
\usepackage[cm]{fullpage}
\usepackage{fancyhdr}
\usepackage{textcomp}
\usepackage{graphicx}
\usepackage{commath}
\usepackage[portuguese]{babel}

\begin{document}

\title{Pré-relatório 3 do Laboratório de Dispositivos e Circuitos Eletrônicos}
\author{Cristiano Silva Júnior: 13/0070629}
\date{\today}
\maketitle

\section{Exercício 1}

\section{Exercício 2}

\section{Exercício 3}

\section{Exercício 4}

\section{Exercício 5}

\section{Exercício 6}

\section{Exercício 7}

\section{Exercício 8}

\section{Exercício 9}

O plágio é considerado um roubo, por ser a apropriação de uma propriedade intelectual.
No caso, se for desejado utilizar o conteúdo intelectual produzido por um terceiro,
devemos citá-lo de maneira adequada. Desta forma, estamos dando crédito ao real dono
daquela produção e estaremos contribuindo com o desenvolvimento científico. A falta de
uma citação implica que nós seríamos os autores daquele texto.

\section{Exercício 10}

A \textit{ABNT} (Associação Brasileira de Normas Técnicas) é uma sociedade privada e
sem fins lucrativos que visa normatizar a produção técnica e intelectual no Brasil por
meio de normas técnicas. No caso, se queremos lançar um produto ou publicar um artigo
neste país, devemos seguir um padrão determinados por comitês especializados pela ABNT
para que haja um denominador comum e que os projetos e os projetistas possam dialogar
entre si.

\section{Referência Bibliográfica}

\begin{itemize}
    \item Artigo sobre diodo do IEEE.
    \item Datasheet do 1NXYZ.
    \item Notas de aula do professor Geovanny.
\end{itemize}

\end{document}
