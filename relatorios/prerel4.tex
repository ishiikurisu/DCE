\documentclass[12pt, a4paper, twoside]{article}
\usepackage[utf8]{inputenc}
\usepackage[cm]{fullpage}
\usepackage{fancyhdr}
\usepackage{textcomp}
\usepackage{graphicx}
\usepackage{commath}
\usepackage[portuguese]{babel}
\usepackage{float}
\usepackage{hyperref}

\begin{document}

\title{Pré-relatório 4 do Laboratório de Dispositivos e Circuitos Eletrônicos}
\author{Cristiano Silva Júnior: 13/0070629}
\date{\today}
\maketitle

Neste relatório, vamos utilizar três modelos para o diodo. O primeiro deles é o modelo ideal, em que o diodo é um circuito fechado para quedas de tensão positivas e um circuito aberto para quedas de tensão negativas.

\begin{figure}[H]
    \centering
    \includegraphics[width=0.4\textwidth]{figs/diode.png}
    \caption{Modelo ideal do diodo}
\end{figure}

O segundo modelo a ser utilizado é o modelo de queda de tensão constante, em que o diodo passa a ser um diodo ideal com uma fonte de tensão em série. Neste caso, o diodo somente conduz para tensões maiores do que a sua tensão de polarização.

\begin{figure}[H]
    \centering
    \includegraphics[width=0.4\textwidth]{figs/diode2.png}
    \caption{Modelo de tensão constante do diodo}
\end{figure}

Neste modelo, podemos levar em conta o efeito Zener, em que, para alguns diodos, o diodo também conduz para quedas de tensão muito negativas. No caso, um diodo com características de Zener conduz também para tensões menores que a sua tensão de Zener.

O terceiro modelo é o chamado diodo real, em que a corrente $i$ que passa pelo diodo depende da tensão $V$ aplicada sobre ele:
$$i = I_s\left( e^{\frac{V}{nV_T}} - 1 \right)$$

\section{Exercício 1}

% TODO Circuito do exercício 1 com modelo atualizado

Utilizando o modelo de tensão constante do diodo para resolver o circuito proposto, podemos atualizar o diagrama, como desenhado na figura 3. Neste caso, é possível que a saída, em regime permanente, será uma tensão DC $v_o = V_p - V_D$, onde $V_D$ é a queda de tensão do diodo.

\section{Exercício 2}

% TODO Adicionar circuito do exercício 2 tanto com condução (a) como sem condução (b) do diodo.

Analisando a corrente $i_L$ da carga no circuito em questão, nota-se que a saída terá uma forma de onda similar a saída, mas com valor máximo $i_{Lmax} = \frac{V_P}{R_L}$. Isto é, ela acompanhará a senoide da entrada quando crescer mas, quando a entrada começar a decair, devido ao capacitor, a tensão no resistor decairá vagarosamente, não chegando a se anular até encontrar uma tensão de crescimento da entrada novamente. Desta forma, a saída é similar a uma tensão DC, com algumas variações.

\section{Exercício 3}

Neste laboratório, utilizaremos o diodo $1N4739$, que é caracterizado por ser um diodo Zener para aplicações de potência. Analisando o \textit{datasheet} do componente fornecido pela \textit{Fairchild Semiconductors}, podemos estudar alguns de seus valores nominais:

\begin{itemize}
    \item Tensão de Zener $V_Z = 9,1V$ a $I_Z = 28ma$;
    \item Máxima impedância $Z_Z = 5 \Omega$;
\end{itemize}

\section{Exercício 4}

% TODO Obter livro da disciplina

\section{Exercício 5}

Para uma aplicação real, o circuito proposto no exercício, que pode ser visto na figura 6 deste relatório, não pode ser utilizado como uma fonte de tensão DC, já que há uma variação muito grande em sua saída para ser considerado uma fonte DC.

Além disso, por questões de segurança, espera-se que a impedância nas saídas de um componente como este seja nula, e logo se nota que a impedância é, considerando o transformador como ideal, $Z = \frac{1}{\jmath C \omega}$, onde $\omega = 120 \pi$ radianos, para o sistema energético brasileiro.

\section{Exercício 6}

Assumindo o transformador utilizado no problema como ideal, espera=se que a sua frequência de saída seja igual à frequência da linha. Neste caso, como a energia elétrica no Brasil é transmitida a $60Hz$, então a frequência de excitação do circuito é de $60Hz$.

\section{Exercício 7}

O valor RMS de uma medida $V$ é definida como
$$ V_{RMS} = \left( \frac{1}{T} \int_0^T V(t)^2 \mathrm{d}t \right)^{\frac{1}{2}} $$
onde $T$ é o período da onda que descreve $V$. No nosso caso,
$$ V(t) = A \sin(2 \pi f t) $$
pois esperamos este formato na energia fornecida pelo sistema energético brasileiro. Substituindo $V$ na definição do valor RMS, notamos que
$$ V_{RMS} = \frac{A}{\sqrt{2}} $$
Como no problema $V_{RMS} = 12V$, então o valor de pico da tensão será
$$ A = \sqrt{2} \cdot V_{RMS} = 16.97V $$

\section{Exercício 8}

A tensão de ondulação (ou de \textit{ripple}) é...

\section{Exercício 9}

$$v_r = IDK$$

\section{Exercício 10}

Pela lei de Ohm, a tensão sobre um componente resistor é $V=RI$. Logo, se espera que, quanto menor a resistência, menor a tensão sobre ele.

\section{Exercício 11}

Eu não sei porque tem um diodo Zener lá.

\section{Exercício 12}

Um circuito regulador tem como premissa a regulação da tensão sobre ele para que ela se mantenha constante. Em geral, ele construído com um circuito retificador seguido de alguns filtros à escolha do fabricante, já que o retificador não tem uma saída constante, como comentado no exercício 5 deste pré-relatório.

\section{Referência Bibliográfica}

\begin{itemize}
    \item ???. 1N4739A: Zener 9.1V 1W 5\%. Disponível em \url{http://www.onsemi.com/PowerSolutions/product.do?id=1N4739A}. Acesso em 25 de Setembro de 2017.
    \item Notas de aula do professor Geovanny.
\end{itemize}

\end{document}
