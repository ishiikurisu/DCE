\documentclass{article}
\usepackage[utf8]{inputenc}

\title{Pré-experimento 6 de LDCE}
\author{Cris Joe Silva Jr.}
\date{\today}

\begin{document}

\maketitle

\section{Introdução}

\section{Questionamentos}

\subsection{Questão 1}

O \textit{BC548} se caracteriza por ser um transistor de uso geral mas com especificações de tensão e corrente grandes o suficiente para ser utilizado como um dispositivo de eletrônica de potência. Já o \textit{BC338} é um transistor com especificações mais próprias para baratear sistemas de sinal, se tornando impróprio como transistor de potência.

\subsection{Questão 2}

O transistor \textit{TIP31} é um transistor NPN, enquanto o transistor \textit{TIP32} é um transistor PNP. Fora isso, todas as suas especificações técnicas são equivalentes. Contudo, eles não podem ser utilizados intercambiavelmente por funcionarem de maneira complementar.

\subsection{Questão 3}

% TODO Adicionar imagem do circuito em análise.

% TODO Adicionar imagem do circuito equivalente.

\subsection{Questão 4}

\subsection{Questão 5}

\subsection{Questão 6}

% TODO Simular circuito.

\subsection{Questão 7}

A saída de um circuito é do tipo \textit{push-pull} quando ela não pode ser dividida com a saída de outros circuitos. É a saída geralmente utilizada em circuitos digitais a fim de facilitar o projeto de um sistema mais complexo.

\section{Referência Bibliográfica}

\begin{itemize}
    \item TODO Add datasheets
\end{itemize}

\end{document}
